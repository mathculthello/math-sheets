\documentclass[12pt]{article}
\usepackage[utf8]{inputenc}
\usepackage[russian]{babel}
\usepackage{amsmath}
\usepackage{setspace}
\usepackage{indentfirst}


\begin{document}
\section*{Бином Ньютона}

Ниже представлены задачи к лекции «Бином Ньютона» к курсу «100 уроков математики» Алексея Владимировича Саватеева. 
Задачи разделены на 2 вида: типовые (их можно решить прямо используя формулу бинома Ньютона) и нетиповые (решение этих задач потребует большего количества времени и «математической смекалки»).

\subsection*{Типовые задачи}

Задача 1. Разложить по формуле бином $(a-\sqrt{2})^6$.

Задача 2. Найти шестой член разложения $(1-2z)^{21}$.

Задача 3. Найдите два средних члена разложения $(a^3+ab)^{21}$.

Задача 4. В биномиальном разложении $\left(x^3+\dfrac{1}{x^3} \right)^{18}$ найти член разложения, не содержащий x.

\subsection*{Нетиповые задачи}

\textit{Определение 1}. Треугольником Паскаля называется треугольная таблица, составленная из чисел по следующему правилу: строка с номером n состоит из n чисел, первое и последнее числа каждой строки равны единице, а каждое из остальных чисел равно сумме двух ближайших к нему чисел предыдущей строки. Число, стоящее на $(k +1)$-м месте $(n+1)$-й строки, обозначается $\dbinom{n}{k}$

	\begin{table}[h]
		\begin{center}
		\begin{tabular}{lllllll}
			&  &  & 1 &  &  &  \\
			&  & 1 &  & 1 &  &  \\
			& 1 &  & 2 &  & 1 &  \\
			1 &  & 3 &  & 3 &  & 1
		\end{tabular}

	\dots
	\end{center}
	\end{table}

Задача 1. Выпишите первые 10 строк треугольника Паскаля.

Задача 2. Запишите в виде $\dbinom{a}{b}$ числа предыдущей строки, ближайшие к числу $\dbinom{n}{m}$.

Задача 3. Докажите, что $\dbinom{n}{m}=\dbinom{n}{n-m}$.

Задача 4. В каких строках треугольника Паскаля все числа нечётные?

\textit{Определение 2}. Числом сочетаний из $n$ по $m$ называется количество $m$-элементных подмножеств множества из $n$ элементов. Обозначение: $C_n^m$.

Задача 5. Найдите: а) $C_{100}^1$, б) $C_4^2$, в) $C_5^2$, г) $C_6^4$.

Задача 6. Раскройте скобки в выражениях $(a+b)$, $(a+b)^2$, $(a+b)^3$, $(a+b)^4$ и выпишите результаты друг под другом.
Обратите внимание, что коэффициенты образуют треугольник Паскаля.
 
Задача 7. Докажите, что: $(a+b)^n = \dbinom{n}{0}a^n+\dbinom{n}{1}a^{n-1}b+\dbinom{n}{2}a^{n-2}b^2+\dots+\dbinom{n}{n}b^n$

Задача 8. Правило Паскаля: $C_n^m=C_{n-1}^m+C_{n-1}^{m-1}$.

Задача 9. Биномиальные коэффициенты членов разложения, равноотстоящих от концов разложения, равны между собой:
$C_n^m=C_n^{n-1}$. (правило симметрии).
 
Задача 10. Докажите, что $\dbinom{n}{m}=C_n^m$.

Задача 11. Сумма биномиальных коэффициентов всех членов разложения равна $2^n$.

Задача 12. Сумма биномиальных коэффициентов, стоящих на нечетных местах, равна сумме биномиальных коэффициентов, стоящих на четных местах и равна $2^{n-1}$.

Задача 13. Любой биномиальный коэффициент, начиная со второго, равен произведению предшествующего биномиального коэффициента и дроби $\dfrac{n-(m-1)}{m}$, т.е. $C_n^m=C_n^{m-1}\cdot\dfrac{n-(m-1)}{m}$.

Задача 14. Докажите, что число способов пройти из левого нижнего угла прямоугольника $(m \times n)$ в правый верхний, двигаясь только вверх или вправо по границам клеток, равно $\dbinom{n+m}{n}$.
	
\end{document}